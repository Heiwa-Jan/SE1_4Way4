\documentclass[a4paper, 12pt]{article}
\usepackage[right = 1.5cm, left = 2.5cm, top = 2cm, bottom = 2cm]{geometry}
\usepackage{fancyhdr}
\usepackage{csquotes}
\usepackage[ngerman]{babel}
\pagestyle{fancy}


\lhead{Testplan 4Way4 Team Jan}
\rhead{\today}
\begin{document}

	\vspace*{\fill}
	%Title
	{\centering \section*{{\underline{\Huge Testplan 4Way4}}}}\vspace{4ex}
	\section*{Testplan von Lorenz Schwab und Lukas Koddenberg}
	\vfill
	
	\pagebreak
	\section{Testgegenstand}
	Testgegenstand ist das in der SE1-Vorlesung vorgestellte Spiel \enquote{4\,Way\,4}, welches von jedem
	Team programmiert werden soll. Das Spiel kann als eine Variante des beliebten Brettspielklassikers
	\enquote{vier gewinnt} angesehen werden.
	
	\section{Umfang}
	Zu testen sind folgende funktionale und nicht funktionale Anforderungen:
	\begin{itemize}
		\item Spielfeldgröße von $7*7$ bis $10*10$. Es muss nicht quadratisch sein.
		\item jeder Spielstein (Token) belegt genau ein Feld.
		\item Spielmodus Spieler\,1 gegen Spieler\,2
		\item Die Spielregeln werden eingehalten.
		\item Spielmodus Spieler\,1 gegen CPU: \\
		CPU in drei verschiedenen Schwierigkeitsstufen
	\end{itemize}
	
	\section{Spielregeln: }
	\begin{enumerate}
		\item Wird ein Token eingeworfen, so werden alle auf dem Feld befindliche Tokens so weit in
		die Richtung gerückt, in die der Token das Feld betreten hat, wie möglich. Die Spielfeldränder sind
		dabei die Höchstgrenze.
		
		\item Gewonnen hat der Spieler, der als erstes vier Tokens in einer Reihe hat. Eine Reihe kann
		horizontal, vertikal und diagonal sein.
		
		\item Das Spiel endet, wenn ein Spieler oder die CPU gewinnt, oder wenn alle Felder des Spielfeldes
		belegt sind. Dann gewinnt der Spieler, der den letzten Zug machen konnte.
		
		\item Spielmodus Spieler gegen CPU: \\
		Der Spieler entscheidet, wer beginnt.
	\end{enumerate}
	
	\section{Testen der funktionalen Anforderungen}
	\begin{itemize}
		\item Spielfeldgröße von $7*7$ bis $10*10$. Es muss nicht quadratisch sein.
		\item jeder Spielstein (Token) belegt genau ein Feld.
		\item Spielmodus Spieler gegen Spieler
		\item Die Spielregeln werden eingehalten.
	\end{itemize}
	
	\pagebreak
	
	\subsection{Testen der funktionalen Anforderungen}
	\subsubsection{Testen auf dem Papier (Black\,-\,Box\,-\,Test)}
	Die ersten Tests wurden von den Teammitgliedern mit Stift und Papier \\
	(meist auf einem $7*7$ - Spielfeld) durchgeführt, um die zugrundeliegende Funktionsweise	
	des Spiels zu verstehen. Außerdem wurde eine Runde mit dem Kunden zusammen an der Tafel gespielt, 
	um Verständnisfragen zu klären.
	
	\subsubsection{JUNIT}
	Die meisten Tests sind aus Gründen der Automatisierung mit JUNIT geschrieben, denn bei über 90\% der Fälle
	macht es einfach keinen Sinn, manuell zu testen.\newline
	Vor allem müssen die Extremfälle getestet werden. Extremfälle sind zum Beispiel das Abfangen von 
	Exceptions bei Falscheingaben, um zu verhindern, dass das Spiel aufgrund einer Falscheingabe abstürzt. 
	
	\subsection{Tests: }
	\subsubsection{Testkriterien}
	Die Spielfeldgröße muss zwischen $7*7$ und $10*10$ liegen.\\
	Das Spielfeld kann auch rechteckig sein, nicht nur quadratisch.\\
	Jeder Token belegt ein Feld.\\
	Spielmodus Spieler\,1 gegen Spieler\,2.\\
	Die Spielregeln werden eingehalten.\\
	Integritätstest: Es werden zwei KIs instantiiert. Diese müssen dann gegeneinander
	spielen. Eventuell auftretende Fehler müssen behoben werden, bis beide KIs 
	problemlos gegeneinander spielen können.
	
	\subsubsection{Testende}
	Die Spielfeldgröße kann im Bereich $7*7$ bis $10*10$ ganzzahlig festgelegt werden.\\
	Das Spielfeld kann rechteckig sein, nicht nur quadratisch.\\
	Jeder Token belegt ein Feld.\\
	Es gibt den Spielmodus Spieler\,1 gegen Spieler\,2.\\
	Die Spielregeln werden eingehalten.\\
	Die KIs können problemlos instantiiert werden und spielen gegeneinander. Eine der
	KIs gewinnt das Spiel. Beide KIs erkennen das Ende des Spiels.
	
	\subsubsection{Testabbruch}
	Die Spielfeldgröße liegt nicht zwischen $7*7$ und $10*10$ oder ist nicht ganzzahlig.\\
	Das Spielfeld kann nur quadratisch sein oder andere Formen als ein Rechteck annehmen.\\
	Jeder Token belegt mehr oder weniger als ein Feld.\\
	Spielmodus Spieler\,1 gegen Spieler\,2 existiert nicht.\\
	Die Spielregeln werden nicht eingehalten.\\		
	Es treten Fehler auf, die das Spiel sofort beenden. Eine KI denkt, das Ende des Spiels
	ist erreicht, aber die andere KI spielt weiter. Sonstige Fehler treten auf und das
	gewünschte Ergebnis wird nicht erzielt.
	
	\subsection{Weitere Tests: }
	
	Ferner müssen noch zu jeder Methode in jeder Klasse Tests erstellt werden, die die 
	ordnungsgemäße Funktion der Methoden sicherstellen. Dies wird folgendermaßen geschehen:
	\begin{enumerate}
		\item Instantiieren der nötigen Objekte
		\item Manipulation der jeweiligen Attribute der Objekte
		\subitem ggf. Herbeiführen von Ausnahmen und behandeln dieser
		\item Abfragen von Zuständen, ausgelösten Ausnahmen etc.
		\item Löschen der Objekte
	\end{enumerate}
	Die obigen Schritte lassen sich für mehrere Variablen durchführen; dadurch müssen 
	auch mehrere dieser doch sehr ähnlichen Tests geschrieben werden. Das Prinzip jedoch
	ändert sich nicht sonderlich. So ergeben sich dann mehrere Testfälle für eine Methode.\\
	Durch die Tests wird versucht, eine möglichst hohe Testabdeckung zu erlangen, wenngleich
	eine 100\%\,-\,ige Testabdeckung nahezu unmöglich ist. Das Ziel von Team Jan ist es,
	eine Testabdeckung von $\geq$ 80\% zu erreichen.\\
	In diesem Dokument werden die Testmethoden, die sich besonders von den obigen 
	Standardtests unterscheiden, aufgeführt:\\
	\subsubsection{Besondere Tests: }
	
	*noch keine besonderen Tests eingefügt*
	
\end{document}